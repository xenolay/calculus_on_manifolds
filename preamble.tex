\usepackage{amsmath}
\usepackage{amsthm}
\usepackage{amsfonts}
\usepackage{mathrsfs}
\usepackage{mathtools}
\usepackage{txfonts}
\usepackage{amscd}
\usepackage[usenames]{color}
\usepackage{thmtools}
\usepackage{braket}
\usepackage{url}
\usepackage{enumitem}
\usepackage{mathtools}
\usepackage{thm-restate}
\usepackage{cleveref}
\mathtoolsset{showonlyrefs,showmanualtags}
\usepackage{setspace}

\newtheoremstyle{mythm}% name
{10pt}% Space above
{15pt}% Space below
{\normalfont}% Body font
{}% Indent amount
{\bfseries}% Theorem head font
{.}% Punctuation after theorem head
{5pt plus 1pt minus 1pt}% Space after theorem head
{}% Theorem head spec (can be left empty, meaning ‘normal’)

% thm:定理番号を振る.
% thm*:定理番号を振らない.declaretheorem を使わない(そのうち直したい).

\theoremstyle{mythm}
\declaretheorem[name=定理,numberwithin=section]{thm}
\declaretheorem[name=定義,numberlike=thm]{defi}
\declaretheorem[name=補題,numberlike=thm]{lem}
\declaretheorem[name=問,numberlike=thm]{que}
\declaretheorem[name=系,numberlike=thm]{cor}
\declaretheorem[name=命題,numberlike=thm]{prop}
\declaretheorem[name=余談,numberlike=thm]{dig}
\declaretheorem[name=主張,numberlike=thm]{state}
\declaretheorem[name=例,numberlike=thm]{exm}
\declaretheorem[name=注意,numberlike=thm]{rem}
\newtheorem*{thm*}{定理}
\newtheorem*{lem*}{補題}
\newtheorem*{defi*}{定義}
\newtheorem*{exm*}{例}
\newtheorem*{cor*}{系}
\newtheorem*{que*}{問}
\newtheorem*{state*}{主張}
\newtheorem*{prop*}{命題}

\declaretheoremstyle[
  spaceabove=0pt, spacebelow=18pt,
  headfont=\bfseries\sffamily,
  notefont=\bfseries\sffamily,
  bodyfont=\normalfont\leftskip1.5em,
  notebraces={(}{)},
  postheadspace=1em,
  numbered=no,
  qed=$\blacksquare$
]{myproof}
\declaretheorem[title=証明, style=myproof]{myproof}
\renewenvironment{proof}{\begin{myproof}}{\end{myproof}}
\renewcommand{\labelenumi}{(\roman{enumi})}

\newcommand{\Opensets}{\mathscr{O}}
\newcommand{\Closedsets}{\mathscr{A}}
\newcommand{\Sym}{\mathfrak{S}}
\newcommand{\Natu}{\mathbb{N}}
\newcommand{\Real}{\mathbb{R}}
\newcommand{\Cpx}{\mathbb{C}}
\newcommand{\Quad}{\mathbb{Q}}
\newcommand{\Tensor}{\mathcal{T}}
\newcommand{\Sphere}{\mathbb{S}}
\newcommand{\GrassAlg}{\bigwedge}
\newcommand{\DiffForm}{\Omega}
\DeclareMathOperator{\diver}{div}
\DeclareMathOperator{\rot}{rot}
\DeclareMathOperator{\grad}{grad}
\DeclareMathOperator{\supp}{supp}
\DeclareMathOperator{\Alt}{Alt}
\DeclareMathOperator{\sgn}{sgn}
\DeclareMathOperator{\Tan}{Tan}
\DeclareMathOperator{\id}{id}
\DeclareMathOperator{\Vect}{Vect}
\newcommand{\dif}{\mathop{}\!d}