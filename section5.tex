\section{鎖体上の微分形式とその積分}

本節では微分形式を定義して Stokes の定理を定式化する.Spivak にも述べてあるように,適切な諸概念を準備した上であれば,Stokes の定理の証明は特に難しくない.裏返せば,証明が当たり前になるように諸概念を用意することが本節の目標である.

\setcounter{subsection}{-1} % いろいろ思うところがあってわざとやっている.
\subsection{微分形式の定義に向けて}

微分形式の定義には種々の代数的な概念が必要になる.のだが,なぜそれらの概念が必要なのかすぐには了解し難いと思われるので,厳密性を抜きにした発見的考察を紹介する.以下の考察は古田幹雄「微分形式と Stokes の定理」(別冊・数理科学「多様体の広がり」,サイエンス社,2008)を参考にした.

このノートの目標が微分可能多様体に対する Stokes の定理であること,この定理が微積分学の基本定理を抽象化したものであることをまえがきで述べた.ところで微分可能多様体とは,大雑把に言えば曲線や曲面の抽象化である.ということで Stokes の定理を理解するということは,曲線や曲面の上で微積分をどう展開するかを理解することと不可分であると言ってよいだろう.

ところで,このノートにおいて積分を定義し,その性質について調べてきた.積分の定義を思い出してみる;単関数$f = \sum a_i 1_{A_i}$の積分は$\int f \dif \mu = \sum a_i \mu(A_i)$で与えられていた.非負可測関数$f \colon X \to [0,\infty]$に対しては,
\begin{equation}
\int f \dif \mu \coloneqq \sup \left\{ \int g \dif \mu \ \middle|  \ 0 \leq g \leq f, g {は単関数} \right\}
\end{equation}によって定義した.即ち,可測関数を単関数によって下から近似することで求めた.単関数によって下から近似するとは,おおらかに言えば,空間$X$をいくつかの可測集合$\set{A_i}$に分割し,各集合$A_i$における近似値$a_i$を定めることであった.直感的には,$X$の分割$A_i$が何かしらの意味で細かくなればなるほど近似の精度が良くなっていくと思われるし,実際に\cref{非負単関数の列}の証明では,$X$の分割を細かくしていくことで,所与の可測関数に収束する単関数の列を作った.ということで,大雑把で感覚的な物言いをすると,$X$の「とても細かい分割」$\set{A_i}$が与えられているとき,$a_i \in A_i$を任意に取れば
\begin{equation}
\int f \dif \mu \approx \sum_{A_i} f(a_i) \mu(A_i)
\end{equation}という近似式が成り立つ,ということができよう.

さて,このような積分の概念を,一般の「曲面」に一般化することを考えよう.一般的な「曲面」の定義を考えるのも骨が折れそうなので,いったん具体的に,$\Real^3$内の単位球面
\begin{equation}
\Sphere^2 \coloneqq \set{ (x^1, x^2, x^3) \in \Real^3 | (x^1)^2 + (x^2)^2 + (x^3)^2 = 1}
\end{equation}を例にして考えることにしたい.もっと即物的に言えば$\Sphere^2$を宇宙$\Real^3$の中にある地球表面のモデルだと思ってもよい\footnote{ある時刻での宇宙が$\Real^3$でモデリングできるかどうかはここでは問題にしない.}.$f \colon \Sphere^2 \to \Real$の「積分」を定義するための道筋を考えることがここでの目標である.$f$に具体的なイメージがほしければ,地表における何かしらの物質の密度分布が与えられていると思って,その「地球での全質量」を求めようとしていると思えばよい.

\begin{que}[*]
もちろん積分を定義したいだけであれば$\Sphere^2$上の「Lebesgue 測度から標準的に定まる」測度が作れればよい.この測度は表面測度と呼ばれる.表面測度の構成だけを目的にするならばこのような発見的考察をしなくとも済むことが知られている.のだが,そのように議論すると微分形式と表面測度の繋がりがわからないままになると思う.
\end{que}

測度空間における(あるいは,Lebesgue 測度に関する)積分のやり方に則れば,適当な部分集合$A \subset \Sphere^2$に対してその「大きさ」$\mu(A)$が定まればよさそうだとわかる.いの一番に考えつくのは,(さながら地図に緯線と経線を引くが如く)$A$に適当な座標軸を引いて$\Real^2$の部分集合と同一視し,$A$に2次元 Lebesgue 測度を入れてしまうことである.のだが,このやり方には2つの問題がある.
\begin{itemize}
\item 測度が座標軸のとり方に依存して定まっているが,現実世界で考えると,長さの基準を決めるのは地図帳の出版社ではなくて光速である.$\Sphere^2$上に適当に入れた座標に依存することがあってはならない.
\item 紙に印刷された世界地図は,実際には長さや面積の情報を正確に反映できていない.そこから類推するに,いきなり大域的な考察をすると何かしらの「歪み」に突き当たりそうなので,考察は局所的にしたほうが安全そうである.
\end{itemize}
ということで,任意の部分集合$A \subset \Sphere^2$を考えるのはやめて,考察を局所的にする.すなわち,ある点$m \in \Sphere^2$のまわりに広がった「うんと小さい領域」$M \subset \Sphere^2$を考えて,$M$上だけで測度を作ろうとすることで,曲面の大域的な歪みを無視できるようにする.また,$M$には座標軸を入れず,ある程度小さなベクトルだけを考えることにして,$m$を原点とする2次元線型空間とみなしてみる.$M$上には(例えば光速から計算された)長さや面積の情報がアプリオリに決まっていると考えて,$m$に足を持つような2つの小さいベクトル$v_m, w_m \in M$の張る平行四辺形の面積を$\omega_m(v_m,w_m)$と書く.やや天下りなのだが,$\omega_m$に見出せそうな性質を列挙しておく;
\begin{itemize}
\item $v,w$は充分小さいので,$\omega_m$は各引数ごとに線型であることにする.
\item 同じ方向を向いた2本のベクトルが張る平行四辺形の面積は0だと思うので,$\omega_m(v_m,v_m) = 0$とする.
\item ベクトルが右手系なのか左手系なのかを込めて,符号つきで面積を測ることにして,$\omega_m(v,w) = - \omega_m(w_m,v_m)$とする.
\end{itemize}
そのように局所的に作った$\omega_m$という量を,いろいろなところで一斉に考えて,関数との値と掛け合わせて足し上げることで,積分の定義が得られるかもしれない.すなわち,いい感じに$\Sphere^2$を細かな領域に分割した上で
\begin{equation}
\int f \omega \approx \sum_{M \subset \Sphere^2} f(m) \cdot \omega_m(v_m, w_m).
\end{equation}として「$f \omega$の積分」を定義しようと試みるのである.実際にはこの方針に基づいた試みは(もちろん適切な,決して短いとは言い難い準備と修正が必要になるものの)驚くことにうまく行ってしまう.特にこの$\omega_m$を適切に定式化することで,曲面における測度の「もと」が得られることがわかっている.この節は$\omega_m$ならびに関係する概念たちをきちんと捉えるまでの道筋を整備することに割かれると言っても過言ではない.

ここまで出てきた概念がどのように厳密化されるのかを列挙して,この項のむすびとする.

\begin{itemize}
\item 「$m$の周りに広がったうんと小さい領域$M$」は,「$m$における接ベクトル空間$T_m \Sphere^2$」となる.$v_m,w_m$は実際には$T_m \Sphere^2$の元と考えることができる.$T_m \Sphere^2$の元を接ベクトルという.
\item 上にあらわれた$\omega_m$を,$m$に関する写像だとみなしたもの$m \mapsto \omega_m$を微分形式という.即ち,接ベクトルをいくつか受け取れる交代的多重線型写像が曲面の各点で定まっているとき,それを微分形式と呼ぶ.ここでの発見的考察に拠って得ようとした「積分」とは,微分形式の積分にほかならない.
\item 「長さや面積の情報がアプリオリに決まっている」という言葉を厳密に述べれば,「多様体$\Sphere^2$上に Riemann 計量が与えられている」となる.「光速から計算された」という文面は,「包含写像$\Sphere^2 \hookrightarrow \Real^3$から定まる誘導計量が与えられた」と言い換えることで表現できる.
\item ここで「曲面」と呼ばれているものは,数学的には微分可能多様体と呼ばれるものの一例である.
\end{itemize}

\subsection{双対空間とテンソル積}

微分形式を定義するのに必要な線型代数の事項として,双対空間とテンソル積について扱う.以下しばらく,$V$ならびに$W$は線型空間とする.

\begin{defi}
$V^* \coloneqq \set{f \colon V \to \Real | f は線型写像 }$を$V$の双対空間という.
\end{defi}

\begin{que}
$V^*$は線型空間の構造を持つ.
\end{que}

\begin{dig}
$\Real^n$が「縦ベクトルからなる空間」だとすれば$(\Real^n)^*$は「横ベクトルからなる空間」であると捉えられないこともない,ということを下支えするのが以下の命題である.ちなみにこの直感は無限次元だと破壊されるので,程々にしておくのが良いらしい…が,私は平気な顔して横ベクトルのことだと思っているフシがある.
\end{dig}

\begin{que}
$x \in \Real^n$に対し,$\varphi_x \in (\Real^n)^*$を$\varphi_x(y) \coloneqq \langle x,y \rangle$で定める.写像$T \colon \Real^n \ni x \mapsto \varphi_x \in (\Real^n)^*$は線型同型写像である.
\end{que}

\begin{que}[Riesz の表現定理,**]
$H$を Hilbert 空間とする.$x \in H$に対し,$\varphi_x \in H^*$を$\varphi_x(y) \coloneqq \langle x,y \rangle$で定める.写像$T \colon H \ni x \mapsto \varphi_x \in H^*$は線型同型写像である.
\end{que}

\begin{que}[**]
$V$を Banach 空間とする.写像$\Phi \colon V \to (V^*)^*$を,$\varphi \in V^*$に対して$\Phi(x)(\varphi) \coloneqq \varphi(x)$で定めると,これは単射な線型写像である.これが同型となるような Banach 空間は反射的であると言われる.ということは世の中には反射的でない Banach 空間が存在する.
\end{que}


\begin{defi}
$V$を線型空間とする.$S \colon V^k \to \Real$が多重線型写像,または$V$上の$k$階テンソルであるとは,各変数に関して$f$が線型写像であること,即ち任意の$1 \leq i \leq k$に対して
\begin{itemize}
\item 任意の$v_1, v_2, \dots, v_k, v_i' \in V$に対して$S(v_1, \dots, v_i, \dots, v_k) + S(v_1, \dots, v_i', \dots, v_k) = S(v_1, \dots, v_i + v_i', \dots, v_k)$
\item 任意の$v_1, v_2, \dots, v_k \in V$ならびに$a \in \Real$に対して$S(v_1, \dots, av_i, \dots, v_k) = aS(v_1, \dots, v_i, \dots, v_k)$
\end{itemize}が成り立つことをいう.$V$上の$k$階テンソル全体の集合を$\Tensor^k(V)$と書く.$\Tensor^1(V)$は$V$の双対空間のことである.
\end{defi}

\begin{que}
$\Tensor^k(V)$は線型空間の構造を持つ.
\end{que}

\begin{dig}
このノートでは Spivak にならって線型空間としてのテンソル積を構成することはせず,多重線型写像としてしか扱わない.実際にはふたつの線型空間$V,W$のテンソル積と呼ばれる線型空間$V \otimes W$を定義して,その空間の元をテンソルと呼ぶのが行儀良いやり方である.のだが,テンソル積は構成がめんどくさいのに加えて,構成よりも普遍性のほうが大事だと言われている(し,私もそう思う).それにこのノートの範囲で普遍性の話を出してもあんまり得るものがなさそうな気がした上,Spivak のとおりにやっても微分形式の定義には困らないので,Spivak のやり方にならうことにした.
\end{dig}

\begin{defi}
$S \in \Tensor^k(V)$および$T \in \Tensor^\ell(V)$に対して,そのテンソル積$S \otimes T \in \Tensor^{k+\ell}(V)$を,
\begin{equation}
S \otimes T(v_1, \dots, v_k, v_{k+1}, \dots, v_{k+\ell}) \coloneqq S(v_1, \dots, v_k) \cdot T(v_{k+1}, \dots, v_{k+\ell})
\end{equation}で定める.
\end{defi}

\begin{que}
$S, S_1, S_2 \in \Tensor^k(V)$および$T, T_1, T_2 \in \Tensor^\ell(V)$,$U \in \Tensor^m(V)$,$a\in\Real$に対して,$(S_1 + S_2) \otimes T = S_1 \otimes T + S_2 \otimes T$,$S \otimes (T_1 + T_2) = S \otimes T_1 + S \otimes T_2$,$(aS) \otimes T = S \otimes (aT) = a(S\otimes T)$,$S\otimes (T \otimes U) = (S \otimes T) \otimes U$が成り立つ.
\end{que}

\subsection{ベクトル場と微分形式}

\subsection{鎖体}

\subsection{微分形式の積分と Stokes の定理}