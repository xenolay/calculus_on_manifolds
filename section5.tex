\section{鎖体上の微分形式とその積分}

本節では微分形式を定義して Stokes の定理を定式化する.Spivak にも述べてあるように,適切な諸概念を準備した上であれば,Stokes の定理の証明は特に難しくない.裏返せば,証明が当たり前になるように諸概念を用意することが本節の目標である.

\setcounter{subsection}{-1} % いろいろ思うところがあってわざとやっている.
\subsection{微分形式の定義に向けて}

微分形式の定義には種々の代数的な概念が必要になる.のだが,なぜそれらの概念が必要なのかすぐには了解し難いと思われるので,厳密性を抜きにした発見的考察を紹介する.以下の考察は古田幹雄「微分形式と Stokes の定理」(別冊・数理科学「多様体の広がり」,サイエンス社,2008)を参考にした.

このノートの目標が微分可能多様体に対する Stokes の定理であること,この定理が微積分学の基本定理を抽象化したものであることをまえがきで述べた.ところで微分可能多様体とは,大雑把に言えば曲線や曲面の抽象化である.ということで Stokes の定理を理解するということは,曲線や曲面の上で微積分をどう展開するかを理解することと不可分であると言ってよいだろう.

ところで,このノートにおいて積分を定義し,その性質について調べてきた.積分の定義を思い出してみる;単関数$f = \sum a_i 1_{A_i}$の積分は$\int f \dif \mu = \sum a_i \mu(A_i)$で与えられていた.非負可測関数$f \colon X \to [0,\infty]$に対しては,
\begin{equation}
\int f \dif \mu \coloneqq \sup \left\{ \int g \dif \mu \ \middle|  \ 0 \leq g \leq f, g {は単関数} \right\}
\end{equation}によって定義した.即ち,可測関数を単関数によって下から近似することで求めた.単関数によって下から近似するとは,おおらかに言えば,空間$X$をいくつかの可測集合$\set{A_i}$に分割し,各集合$A_i$における近似値$a_i$を定めることであった.直感的には,$X$の分割$A_i$が何かしらの意味で細かくなればなるほど近似の精度が良くなっていくと思われるし,実際に\cref{非負単関数の列}の証明では,$X$の分割を細かくしていくことで,所与の可測関数に収束する単関数の列を作った.ということで,大雑把で感覚的な物言いをすると,$X$の「とても細かい分割」$\set{A_i}$が与えられているとき,$a_i \in A_i$を任意に取れば
\begin{equation}
\int f \dif \mu \approx \sum_{A_i} f(a_i) \mu(A_i)
\end{equation}という近似式が成り立つ,ということができよう.

さて,このような積分の概念を,一般の「曲面」に一般化することを考えよう.一般的な「曲面」の定義を考えるのも骨が折れそうなので,いったん具体的に,$\Real^3$内の単位球面
\begin{equation}
\Sphere^2 \coloneqq \set{ (x^1, x^2, x^3) \in \Real^3 | (x^1)^2 + (x^2)^2 + (x^3)^2 = 1}
\end{equation}を例にして考えることにしたい.もっと即物的に言えば$\Sphere^2$を宇宙$\Real^3$の中にある地球表面のモデルだと思ってもよい\footnote{ある時刻での宇宙が$\Real^3$でモデリングできるかどうかはここでは問題にしない.}.$f \colon \Sphere^2 \to \Real$の「積分」を定義するための道筋を考えることがここでの目標である.$f$に具体的なイメージがほしければ,地表における何かしらの物質の密度分布が与えられていると思って,その「地球での全質量」を求めようとしていると思えばよい.

\begin{que}[*]
もちろん積分を定義したいだけであれば$\Sphere^2$上の「Lebesgue 測度から標準的に定まる」測度が作れればよい.この測度は表面測度と呼ばれる.表面測度の構成だけを目的にするならばこのような発見的考察をしなくとも済むことが知られている.のだが,そのように議論すると微分形式と表面測度の繋がりがわからないままになると思う.
\end{que}

測度空間における(あるいは,Lebesgue 測度に関する)積分のやり方に則れば,適当な部分集合$A \subset \Sphere^2$に対してその「大きさ」$\mu(A)$が定まればよさそうだとわかる.いの一番に考えつくのは,(さながら地図に緯線と経線を引くが如く)$A$に適当な座標軸を引いて$\Real^2$の部分集合と同一視し,$A$に2次元 Lebesgue 測度を入れてしまうことである.のだが,このやり方には2つの問題がある.
\begin{itemize}
\item 測度が座標軸のとり方に依存して定まっているが,現実世界で考えると,長さの基準を決めるのは地図帳の出版社ではなくて光速である.$\Sphere^2$上に適当に入れた座標に依存することがあってはならない.
\item 紙に印刷された世界地図は,実際には長さや面積の情報を正確に反映できていない.そこから類推するに,いきなり大域的な考察をすると何かしらの「歪み」に突き当たりそうなので,考察は局所的にしたほうが安全そうである.
\end{itemize}
ということで,任意の部分集合$A \subset \Sphere^2$を考えるのはやめて,考察を局所的にする.すなわち,ある点$m \in \Sphere^2$のまわりに広がった「うんと小さい領域」$M \subset \Sphere^2$を考えて,$M$上だけで測度を作ろうとすることで,曲面の大域的な歪みを無視できるようにする.また,$M$には座標軸を入れず,ある程度小さなベクトルだけを考えることにして,$m$を原点とする2次元線型空間とみなしてみる.$M$上には(例えば光速から計算された)長さや面積の情報がアプリオリに決まっていると考えて,$m$に足を持つような2つの小さいベクトル$v_m, w_m \in M$の張る平行四辺形の面積を$\omega_m(v_m,w_m)$と書く.やや天下りなのだが,$\omega_m$に見出せそうな性質を列挙しておく;
\begin{itemize}
\item $v,w$は充分小さいので,$\omega_m$は各引数ごとに線型であることにする.
\item 同じ方向を向いた2本のベクトルが張る平行四辺形の面積は0だと思うので,$\omega_m(v_m,v_m) = 0$とする.
\item ベクトルが右手系なのか左手系なのかを込めて,符号つきで面積を測ることにして,$\omega_m(v,w) = - \omega_m(w_m,v_m)$とする.
\end{itemize}
そのように局所的に作った$\omega_m$という量を,いろいろなところで一斉に考えて,関数との値と掛け合わせて足し上げることで,積分の定義が得られるかもしれない.すなわち,いい感じに$\Sphere^2$を細かな領域に分割した上で
\begin{equation}
\int f \omega \approx \sum_{M \subset \Sphere^2} f(m) \cdot \omega_m(v_m, w_m).
\end{equation}として「$f \omega$の積分」を定義しようと試みるのである.実際にはこの方針に基づいた試みは(もちろん適切な,決して短いとは言い難い準備と修正が必要になるものの)驚くことにうまく行ってしまう.特にこの$\omega_m$を適切に定式化することで,曲面における測度の「もと」が得られることがわかっている.この節は$\omega_m$ならびに関係する概念たちをきちんと捉えるまでの道筋を整備することに割かれると言っても過言ではない.

ここまで出てきた概念がどのように厳密化されるのかを列挙して,この項のむすびとする.

\begin{itemize}
\item 「$m$の周りに広がったうんと小さい領域$M$」は,「$m$における接ベクトル空間$T_m \Sphere^2$」となる.$v_m,w_m$は実際には$T_m \Sphere^2$の元と考えることができる.$T_m \Sphere^2$の元を接ベクトルという.
\item 上にあらわれた$\omega_m$を,$m$に関する写像だとみなしたもの$m \mapsto \omega_m$を微分形式という.即ち,接ベクトルをいくつか受け取れる交代的多重線型写像が曲面の各点で定まっているとき,それを微分形式と呼ぶ.ここでの発見的考察に拠って得ようとした「積分」とは,微分形式の積分にほかならない.
\item 「長さや面積の情報がアプリオリに決まっている」という言葉を厳密に述べれば,「多様体$\Sphere^2$上に Riemann 計量が与えられている」となる.「光速から計算された」という文面は,「包含写像$\Sphere^2 \hookrightarrow \Real^3$から定まる誘導計量が与えられた」と言い換えることで表現できる.
\item ここで「曲面」と呼ばれているものは,数学的には微分可能多様体と呼ばれるものの一例である.
\end{itemize}

\subsection{双対空間とテンソル積}

微分形式を定義するのに必要な線型代数の事項として,双対空間とテンソル積について扱う.以下しばらく,$V$ならびに$W$は線型空間とする.

\begin{defi}
$V^* \coloneqq \set{f \colon V \to \Real | f は線型写像 }$を$V$の双対空間という.
\end{defi}

\begin{que}
$V^*$は線型空間の構造を持つ.やるだけでめんどくさいので証明を省いてます.
\end{que}

\begin{defi}
$\set{e_1, \dots, e_n}$を$V$の基底とする.各$e_i$($1 \leq i \leq n$)に対し,線型写像$f^i \colon V \to \Real$を,$f^i(e_j) = \delta^i_j$で定める.但し,右辺は Kronecker のデルタ.このとき,$\set{f^1, \dots, f^n}$は$V^*$の基底となる.この基底を$\set{e_1, \dots, e_n}$の双対基底という.
\end{defi}

$\Real^n$が「縦ベクトルからなる空間」だとすれば$(\Real^n)^*$は「横ベクトルからなる空間」であると捉えられないこともない,ということを下支えするのが以下の命題である.

\begin{que}
$x \in \Real^n$に対し,$\varphi_x \in (\Real^n)^*$を$\varphi_x(y) \coloneqq \langle x,y \rangle$で定める.写像$T \colon \Real^n \ni x \mapsto \varphi_x \in (\Real^n)^*$は線型同型写像である.ちなみにこの同型写像は標準内積に依存しているので,特に基底に依存して定まる同型写像である.その意味で「自然ではない」という言い方がされることがあるが,私は自然性のことをすべて忘れました.
\end{que}

\begin{dig}
ちなみにこの直感は無限次元だと破壊されるので,程々にしておくのが良いらしい…が,私は平気な顔して横ベクトルのことだと思っているフシがある.参考までに,広く知られている以下2点の事実を列挙しておく.ちなみに私は証明を読んだことはない.
\end{dig}

\begin{que}[Riesz の表現定理,**]
$H$を Hilbert 空間とする.$x \in H$に対し,$\varphi_x \in H^*$を$\varphi_x(y) \coloneqq \langle x,y \rangle$で定める.写像$T \colon H \ni x \mapsto \varphi_x \in H^*$は線型同型写像である.
\end{que}

\begin{que}[**]
$V$を Banach 空間とする.写像$\Phi \colon V \to (V^*)^*$を,$\varphi \in V^*$に対して$\Phi(x)(\varphi) \coloneqq \varphi(x)$で定めると,これは単射な線型写像である.これが同型となるような Banach 空間は反射的であると言われる.ということは世の中には反射的でない Banach 空間が存在する.
\end{que}


\begin{defi}
$V$を線型空間とする.$S \colon V^k \to \Real$が多重線型写像,または$V$上の$k$階テンソルであるとは,各変数に関して$f$が線型写像であること,即ち任意の$1 \leq i \leq k$に対して
\begin{itemize}
\item 任意の$v_1, v_2, \dots, v_k, v_i' \in V$に対して$S(v_1, \dots, v_i, \dots, v_k) + S(v_1, \dots, v_i', \dots, v_k) = S(v_1, \dots, v_i + v_i', \dots, v_k)$
\item 任意の$v_1, v_2, \dots, v_k \in V$ならびに$a \in \Real$に対して$S(v_1, \dots, av_i, \dots, v_k) = aS(v_1, \dots, v_i, \dots, v_k)$
\end{itemize}が成り立つことをいう.$V$上の$k$階テンソル全体の集合を$\Tensor^k(V)$と書く.$\Tensor^1(V)$は$V$の双対空間のことである.
\end{defi}

\begin{que}
$\Tensor^k(V)$は線型空間の構造を持つ.やるだけでめんどくさいので証明を省いてます.
\end{que}

\begin{dig}
このノートでは Spivak にならって線型空間としてのテンソル積を構成することはせず,多重線型写像としてしか扱わない.実際にはふたつの線型空間$V,W$のテンソル積と呼ばれる線型空間$V \otimes W$を定義して,その空間の元をテンソルと呼ぶのが行儀良いやり方である.のだが,テンソル積は構成がめんどくさいのに加えて,構成よりも普遍性のほうが大事だと言われている(し,私もそう思う).それにこのノートの範囲で普遍性の話を出してもあんまり得るものがなさそうな気がした上,Spivak のとおりにやっても Riemann 計量や微分形式の定義には困らない.なので,Spivak のやり方にならうことにした.
\end{dig}

\begin{defi}
$S \in \Tensor^k(V)$および$T \in \Tensor^\ell(V)$に対して,そのテンソル積$S \otimes T \in \Tensor^{k+\ell}(V)$を,
\begin{equation}
S \otimes T(v_1, \dots, v_k, v_{k+1}, \dots, v_{k+\ell}) \coloneqq S(v_1, \dots, v_k) \cdot T(v_{k+1}, \dots, v_{k+\ell})
\end{equation}で定める.
\end{defi}

\begin{que}
$\otimes$は双線型かつ結合的である.すなわち,$S, S_1, S_2 \in \Tensor^k(V)$および$T, T_1, T_2 \in \Tensor^\ell(V)$,$U \in \Tensor^m(V)$,$a\in\Real$に対して,
\begin{itemize}
\item $(S_1 + S_2) \otimes T = S_1 \otimes T + S_2 \otimes T$
\item $S \otimes (T_1 + T_2) = S \otimes T_1 + S \otimes T_2$
\item $(aS) \otimes T = S \otimes (aT) = a(S\otimes T)$
\item $S\otimes (T \otimes U) = (S \otimes T) \otimes U$
\end{itemize}
が成り立つ.やるだけでめんどくさいので証明を省いてます.
\end{que}

\begin{prop}
$\set{e_1, \dots, e_n}$を$V$の基底,$\set{f^1, \dots, f^n}$をその双対基底とする.このとき,
\begin{equation}
\set{ f^{i_1} \otimes \dots \otimes f^{i_k} | 1 \leq i_1, \dots i_k \leq n }
\end{equation}は$\Tensor^k(V)$の基底である.したがって$\Tensor^k(V)$は$n^k$次元である.
\end{prop}

\begin{proof}
$f^{i_1} \otimes \dots \otimes f^{i_k}$たちが一次独立であることと$\Tensor^k(V)$の生成系であることを示せばよい.まず一次独立性を示す.
\begin{equation}
\sum_{1\leq j_1, \dots, j_k \leq n} a_{j_1, \dots, j_k} f^{j_1} \otimes \dots \otimes f^{j_k} = 0
\end{equation}が成り立っていたと仮定する.このとき左辺の線型写像を$(e_{i_1}, \dots, e_{i_k})$($1 \leq i_1, \dots, i_k \leq n$)に作用させれば$a_{i_1, \dots, i_k}=0$が出るので,一次独立性がわかる.

$\Tensor^k(V)$の生成系になっていることを示すために,$T \in \Tensor^k(V)$を任意に一つ取る.$w_1, \dots, w_n \in V$を任意に取り,それを基底$\set{e_1, \dots, e_n}$で展開した係数を$a_i^j$のように書く;$w_i \eqqcolon \sum_{j} a_i^j e_j$.このとき,
\begin{align}
T(w_1, \dots, w_n) &= T\left( \sum_{j_1} a_1^{j_1} e_{j_1}, \dots, \sum_{j_n} a_1^{j_n} e_{j_n} \right) \\
&= \sum_{j_1} a_1^{j_1} T\left( e_{j_1},  \sum_{j_2} a_2^{j_2} e_{j_2}, \dots, \sum_{j_n} a_n^{j_n} e_{j_n} \right) \\
&= \dotsb \\
&= \sum_{j_1} a_1^{j_1} a_2^{j_2} \dots a_n^{j_n}  T\left( e_{j_1}, \dots, e_{j_n} \right) \\
&= \sum_{j_1} a_1^{j_1} a_2^{j_2} \dots a_n^{j_n}  f^{j_1} \otimes \dots \otimes f^{j_n}
\end{align}となるので,$f^{i_1} \otimes \dots \otimes f^{i_k}$たちは$\Tensor^k(V)$の生成系である.
\end{proof}

\begin{defi}
$f \colon V \to W$を線型写像とする.このとき,$f^* \colon \Tensor^k(W) \to \Tensor^k(V)$を,$T \in \Tensor^k(W)$ならびに$v_1, \dots, v_k \in V$に対して$(f^*T) (v_1, \dots, v_k) \coloneqq T(f(v_1), \dots, f(v_n))$と定める.$f^*T$を$f$による$T$の引き戻しという.
\end{defi}

\begin{dig}
$f^*$を何と呼ぶかはやや迷った.多様体上の微分形式に対しては「引き戻し」という言い方をするのが非常に一般的である.のだが,ただの多重線型写像に対して「引き戻し」という言葉遣いをする例はあまり聞かない気がする.とはいえ呼んで悪いことはないだろう.$f^*$を「$f$が引き起こす写像」とか「$f$によって引き起こされた写像」ということも多い.ちなみに$k=1$の場合は$f^*$を$f$の双対写像というのだが,$k > 1$の場合まで双対写像と呼んでいる人は見たことがない.見たことがないだけで,いるかも知れない.
\end{dig}



\begin{exm}手頃な,そして重要なテンソルの例は内積と行列式である.
\begin{itemize}
\item $\Real^n$の標準内積$\langle \cdot, \cdot \rangle \colon \Real^n \times \Real^n \to \Real$は$\Tensor^2(\Real^n)$の元である.ところで,内積は単にテンソルであるというだけではなくて引数入れ替えに関して対称という性質がある.
\item 行列式$A \mapsto \det A$を,ベクトルを$n$個受け取る多重線型写像$\det \colon \Real^n \times \Real^n \dots \times \Real^n \to \Real$とみなせば,$\det \in \Tensor^n(\Real^n)$である.ところで,行列式は単にテンソルであるというだけではなくて引数入れ替えをすると一定のルールのもとで符号が入れ替わるという性質がある.
\end{itemize}
\end{exm}

直前の例で,それぞれは単にテンソルであるだけではないということを述べた.これらの「追加的に充たされている」性質を抽出して定義されるのが対称テンソルや交代テンソルの概念である.

\begin{defi}
\leavevmode
\begin{itemize}
\item $T \in \Tensor^2(V)$が非退化であるとは,任意の$v \in V$に対して$T(v,v) \neq 0$であることをいう.$T \in \Tensor^2(V)$が正定値であるとは,任意の$v \in V$に対して$T(v,v) > 0$となることをいう.
\item $S \in \Tensor^k(V)$が対称テンソルであるとは,任意の$v_1, \dots, v_k \in V$ならびに$i \neq j$に対して,
\begin{equation}
S(v_1,\dots, v_i, \dots, v_j, \dots, v_k) = S(v_1, \dots, v_j, \dots, v_i, \dots, v_k)
\end{equation}
が成り立つことをいう.$V$上の正定値対称テンソルのことを$V$の内積という.
\item $\omega \in \Tensor^k(V)$が交代テンソルであるとは,任意の$v_1, \dots, v_k \in V$ならびに$i \neq j$に対して,
\begin{equation}
\omega(v_1,\dots, v_i, \dots, v_j, \dots, v_k) = -\omega(v_1, \dots, v_j, \dots, v_i, \dots, v_k)
\end{equation}
が成り立つことをいう.$V$上の$k$階交代テンソル全体の集合を$\Omega^k(V)$で表す.これは$\Tensor^k(V)$の部分空間である.ただし,$k=1$の場合は$\Omega^1(V) = \Tensor^1(V)$である\footnote{$i \neq j$となるような$i,j$が取れないため,交代テンソルが充たすべき条件は常に空虚に真である.}.
\end{itemize}
\end{defi}

特に交代テンソルは微分形式の定義に直結するので,この項で基本的性質を調べておくことにする.それに先立ち,線型代数で行列式を学んだ際に触れたであろう次の事実を思い出しておこう;

\begin{prop}
$F \colon M_n(\Real) \to \Real$を,$\Real^n$の元を$n$個受け取る写像とみなす;$F \colon \Real^n \times \dots \times \Real^n \to \Real$.$F$が交代的な多重線型写像であるならば,ある定数$c  \in \Real$が存在して$F = c \cdot \det$が成り立つ.
\end{prop}

この事実を言い換えると,$F \in \Omega^n(\Real^n)$ならば,$F$は$\det$の定数倍になっているということである.これをさらに言い換えれば,$F$は線型空間として1次元である(基底として$\det$が取れる)ということになる.ここでは,より一般的な状況を調べておく;$n$次元線型空間$V$に対して$\Omega^k(V)$の基底を列挙することで,その次元が$_nC_k = \frac{n!}{k!(n-k)!}$であることを示す.そのためには外積と呼ばれる操作が必要になる.

\begin{defi}
\leavevmode
\begin{itemize}
\item $T \in \Tensor^k(V)$に対して,$\Alt(T) \in \Tensor^k(V)$を
\begin{equation}
\Alt(T)(v_1, \dots, v_k) \coloneqq \frac{1}{k!} \sum_{\sigma \in \Sym_k} (\sgn \sigma )T(v_{\sigma(1)}, \dots, v_{\sigma(k)})
\end{equation}で定める.但し,$\Sym_k$は$k$次対称群\footnote{$[k] \coloneqq \set{1, \cdots, k}$と置いて,$\Sym_k \coloneqq \set{ \sigma \colon [k] \to [k] | \sigma は全単射}$を$k$次対称群という.「群」という文字を見たことがあるかどうかはともかくとして,この概念自体は行列式の定義で見かけていると思う.}.$\Alt \colon \Tensor^k(V) \to \Tensor^k(V)$は線型写像である.すぐ後に見るが,$\Alt(T) \in \Omega^k(V)$となることがわかるので,それを踏まえて$\Alt \colon \Tensor^k(V) \to \Omega^k(V)$を交代化作用素という.頭についている係数のいわれは後でわかる.
\item $\omega \in \Omega^k(V)$,$\eta \in \Omega^\ell(V)$に対して,その外積$\omega \wedge \eta \in \Omega^{k + \ell}(V)$を,
\begin{equation}
\omega \wedge \eta \coloneqq \frac{(k+\ell)!}{k!\ell!} \Alt (\omega \otimes \eta)
\end{equation}で定める.頭についている係数のいわれは Spivak いわく後でわかるらしい.
\end{itemize}
\end{defi}

\begin{que}
$\omega \in \Omega^k(V)$,$\eta \in \Omega^\ell(V)$に対して,$\omega \otimes \eta \in \Omega^{k + \ell}(V)$とは限らない.
\end{que}

\begin{dig}
交代化作用素と外積の定義を天下りに与えたくない気持ちがあるのだが,あまり良い説明を思いつかない.確かにテンソルを交代テンソルに化かすためには(定数倍を除いて)交代化作用素以上に簡単な方法はないとも思うし,交代性を保つような演算として外積より簡単なものを持ってこいと言われても(少なくとも私は)困るしかない.そんなわけで,「これは自然に思いつく」と言われたとしても,渋い顔をしながら「そうかもしれないですね」と言う以外にないのが正直なところである.
\end{dig}

\begin{prop}[交代化作用素の基本性質]\label{交代化作用素の基本性質}
\leavevmode
\begin{enumerate}
\item $\omega \in \Omega^k(V)$,$\sigma \in \Sym_k$ならびに$v_1, \dots, v_k \in V$に対して$\omega(v_{\sigma(1)},\dots,v_{\sigma(k)}) = (\sgn \sigma)\cdot \omega(v_1, \dots, v_k)$.
\item $T \in \Tensor^k(V)$に対して,$\Alt(T) \in \Omega^k(V)$.
\item $\omega \in \Omega^k(V)$に対して,$\Alt(\omega) = \omega$.
\item $T \in \Tensor^k(V)$に対して,$\Alt(\Alt(T)) = \Alt(T)$.
\end{enumerate}
\end{prop}

\begin{proof}
\leavevmode
\begin{enumerate}
\item $\sigma$が互換の場合は交代テンソルの定義そのものである.一般の置換は互換の積に分解できることからよい.
\item $i$と$j$のみを動かしその他を動かさない互換を$(i \ j)$と書くことにして,$\sigma \in \Sym_k$に対し$\sigma' \coloneqq \sigma \circ (i \ j)$と置く.写像$\sigma \mapsto \sigma'$は単射であることに注意すると,$v_1, \dots, v_k \in V$に対して,
\begin{align}
\Alt(T)(v_1, \dots, v_j, \dots, v_i, \dots, v_k) &= \frac{1}{k!} \sum_{\sigma \in \Sym_k} (\sgn \sigma) T(v_{\sigma(1)}, \dots, v_{\sigma(j)}, \dots, v_{\sigma(i)}, \dots, v_{\sigma(k)}) \\
&= \frac{1}{k!} \sum_{\sigma \in \Sym_k} (\sgn\sigma)T(v_{\sigma'(1)}, \dots, v_{\sigma'(i)}, \dots, v_{\sigma'(j)}, \dots, v_{\sigma'(k)}) \\
&= \frac{1}{k!} \sum_{\sigma \in \Sym_k} -(\sgn\sigma')T(v_{\sigma'(1)}, \dots, v_{\sigma'(i)}, \dots, v_{\sigma'(j)}, \dots, v_{\sigma'(k)}) \\
&= \frac{1}{k!} \sum_{\sigma' \in \Sym_k} -(\sgn\sigma')T(v_{\sigma'(1)}, \dots, v_{\sigma'(i)}, \dots, v_{\sigma'(j)}, \dots, v_{\sigma'(k)}) \\
&= -\Alt(T)(v_1, \dots, v_i, \dots, v_j, \dots, v_k)
\end{align}となるからよい.
\item (i) の結果より
\begin{align}
\Alt(\omega) (v_1, \dots, v_k) &= \frac{1}{k!} \sum_{\sigma \in \Sym_k} (\sgn \sigma )\omega(v_{\sigma(1)}, \dots, v_{\sigma(k)}) \\
&= \frac{1}{k!} \sum_{\sigma \in \Sym_k} (\sgn \sigma )(\sgn \sigma)\omega(v_1, \dots, v_k) \\
&= \frac{1}{k!} \sum_{\sigma \in \Sym_k} \omega(v_1, \dots, v_k) \\
&= \omega(v_1,\dots,v_k).
\end{align}
\item (ii) と (iii) よりよい.
\end{enumerate}
\end{proof}

\begin{dig}
証明をよく見ればわかるように,交代化作用素に係数$\frac{1}{k!}$がついているのは$\omega \in \Omega^k(V)$に対して$\Alt(\omega) = \omega$を成り立たせるため,別の言い方をすれば$\Alt$が射影作用素としてはたらく,冪等になる($\Alt^2 = \Alt$が成り立つ)ようにするためである.
\end{dig}

\begin{prop}[外積の基本性質]
\leavevmode
\begin{enumerate}
\item 外積を取る操作は双線型である.即ち,$\omega, \omega_1, \omega_2 \in \Omega^k(V)$ならびに$\eta, \eta_1, \eta_2 \in \Omega^\ell(V)$,$a \in \Real$を任意に取れば,
\begin{itemize}
\item $\omega \wedge (\eta_1 + \eta_2) = \omega \wedge \eta_1 + \omega \wedge \eta_2$
\item $(\omega_1 + \omega_2) \wedge \eta = \omega_1 \wedge \eta + \omega_2 \wedge \eta_2$
\item $a(\omega \wedge \eta) = (a\omega) \wedge \eta = \omega \wedge (a \eta)$
\end{itemize}が成り立つ.
\item 外積を取る操作は次の意味で引き戻しと可換である;$\omega \in \Omega^k(V)$ならびに$\eta \in \Omega^\ell(V)$として,$f \colon V \to W$を線型写像とするとき,$f^*(\omega \wedge \eta) = f^*(\omega) \wedge f^*(\eta)$.
\item 外積を取る操作は次の意味で交代的である;$\omega \in \Omega^k(V)$ならびに$\eta \in \Omega^\ell(V)$に対して$\omega \wedge \eta = (-1)^{k\ell}\eta \wedge \omega$.
\item 外積を取る操作は結合的である.特に,$\omega \in \Omega^k(V)$,$\eta \in \Omega^{\ell}(V)$,$\theta \in \Omega^m(V)$に対して,$(\omega \wedge \eta) \wedge \theta = \omega \wedge(\eta \wedge \theta) = \frac{(k+\ell+m)!}{k! \ell! m!} \Alt(\omega \otimes \eta \otimes \theta)$が成り立つ.
\end{enumerate}
\end{prop}

\begin{proof}
\leavevmode
\begin{enumerate}
\item $\Alt$の線型性と$\otimes$の双線型性よりよい.細かいことはめんどくさいのでさぼりました.
\item めんどくさいのでさぼりました.直接計算すれば (i) よりも簡単に証明できるはず.
\item $\tau \in \Sym_{k + \ell}$を,
\begin{equation}
\tau \coloneqq 
    \begin{pmatrix}
    1 & 2 & \cdots & \ell & \ell+1 & \cdots & \ell + k \\
    k+1 & k+2 & \cdots & k + \ell & 1 & \cdots & k
    \end{pmatrix}
\end{equation}で定めると,$\sgn \tau = (-1)^{k \ell}$である\footnote{いわゆる「あみだくじ」を書いて符号を計算するのが一番簡単だと思う.}.\cref{交代化作用素の基本性質} (i) を繰り返し使うと,
\begin{align}
(\omega \wedge \eta)(v_1,\dots, v_{k+\ell}) &= \frac{1}{(k+\ell)!} \sum_{\sigma \in \Sym_{k+ \ell}} (\sgn \sigma) \omega(v_{\sigma(1)}, \dots, v_{\sigma(k)}) \eta (v_{\sigma(k+1)}, \dots, v_{\sigma(k+\ell)})\\
&= \frac{1}{(k+\ell)!} \sum_{\sigma \in \Sym_{k+ \ell}} (\sgn \sigma)^3 \omega(v_{1}, \dots, v_{k}) \eta (v_{k+1}, \dots, v_{k+\ell})\\
&= \frac{1}{(k+\ell)!} \sum_{\sigma \in \Sym_{k+ \ell}} (\sgn \sigma) \omega(v_{1}, \dots, v_{k}) \eta (v_{k+1}, \dots, v_{k+\ell})\\
&= \frac{1}{(k+\ell)!} \sum_{\sigma \in \Sym_{k+ \ell}} (\sgn \sigma) \omega(v_{\tau(\ell+ 1)}, \dots, v_{\tau(\ell + k)}) \eta (v_{\tau(1)}, \dots, v_{\tau(\ell)})\\
&= \frac{1}{(k+\ell)!} \sum_{\sigma \in \Sym_{k+ \ell}} (\sgn \sigma)(\sgn \tau) \omega(v_{\ell+ 1}, \dots, v_{\ell + k}) \eta (v_{1}, \dots, v_{\ell})\\
&= \frac{(-1)^{k\ell}}{(k+\ell)!} \sum_{\sigma \in \Sym_{k+ \ell}} (\sgn \sigma) \eta (v_{1}, \dots, v_{\ell})\omega(v_{\ell+ 1}, \dots, v_{\ell + k}) \\
&= (-1)^{k\ell}(\eta \wedge \omega)(v_1,\dots, v_{k+\ell}).
\end{align}
\item 証明を3段階に分割する.
\begin{enumerate}
\item $S \in \Tensor^k(V)$並びに$T \in \Tensor^\ell(V)$に対して$\Alt(S)=0$または$\Alt(T)=0$ならば$\Alt(S \otimes T)=0$となることを示す.議論は同様なので$\Alt(S)=0$の場合のみ証明する.ここでは$\Sym_k$を($\sigma \in \Sym_k$は$k+1, \dots, k+\ell$を動かさないような$\Sym_{k+\ell}$の元だとみなすことで)$\Sym_{k+\ell}$の部分集合とみなす.
\begin{align}
\sum_{\sigma \in \Sym_k} S(v_{\sigma(1)}, \dots, v_{\sigma(k)})T(v_{\sigma(k+1)}, \dots, v_{\sigma(k + \ell)}) &= \sum_{\sigma \in \Sym_k} S(v_{\sigma(1)}, \dots, v_{\sigma(k)})T(v_{k+1}, \dots, v_{k + \ell}) \\
&= \left(  \sum_{\sigma \in \Sym_k} S(v_{\sigma(1)}, \dots, v_{\sigma(k)}) \right) T(v_{k+1}, \dots, v_{k + \ell}) \\
&=0.
\end{align}
次いで,$\tau \in \Sym_{k+\ell} \setminus \Sym_k$を任意に取る\footnote{ここから先の議論は群の剰余類の考え方を知っていれば見通しがよいと思う.要するに$\Sym_{k+\ell}$を部分群$\Sym_k$の定める右剰余類に分解し,各剰余類において和が0だから全体の和も0であるということを(群の言葉を一切出さずに)議論しているに過ぎない.}.$\Sym_k \tau \coloneqq \set{\sigma \tau | \sigma \in \Sym_k}$と置くと,
\begin{align}
\sum_{\sigma \in \Sym_k \tau} S(v_{\sigma(1)}, \dots, v_{\sigma(k)})T(v_{\sigma(k+1)}, \dots, v_{\sigma(k + \ell)}) &= \sum_{\sigma' \in \Sym_k} S(v_{\sigma'\tau(1)}, \dots, v_{\sigma'\tau(k)})T(v_{\tau(k+1)}, \dots, v_{\tau(k + \ell)}) \\
&= \left(\sum_{\sigma' \in \Sym_k} S(v_{\sigma'\tau(1)}, \dots, v_{\sigma'\tau(k)}) \right) T(v_{\tau(k+1)}, \dots, v_{\tau(k + \ell)}) \\
&= \left(\sum_{\sigma' \in \Sym_k} S(v_{\sigma'(1)}, \dots, v_{\sigma'(k)}) \right) T(v_{\tau(k+1)}, \dots, v_{\tau(k + \ell)}) \\
&=0.
\end{align}ここで,$\Sym_k \cap \Sym_k \tau = \emptyset$が次のようにしてわかる.そうでないとして,$\sigma_0 \in \Sym_k \cap \Sym_k \tau$を取ると,$\sigma_0 = \sigma_1 \tau$となるような$\sigma_1 \in \Sym_k$がある.したがって$\tau = (\sigma_1)^{-1}\sigma_0 \in \Sym_k$となるが,これは$\tau$の定め方に反する.同様に,相異なる$\tau_1, \tau_2 \in \Sym_{k+\ell} \setminus \Sym_k$を取ると$\Sym_k \tau_1 \cap \Sym_k \tau_2 = \emptyset$であることもわかる.以上より,総和$\sum_{\sigma \in \Sym_{k+\ell}}$は$\sum_{\sigma \in \Sym_k} + \sum_{\tau \in \Sym_{k+\ell} \setminus \Sym_k} \sum_{\sigma \in \Sym_k \tau}$の形に分解できる.分解した各々の項が0であることはすでに見たので,全体の総和も0である.
\item $S \in \Tensor^k(V), T \in \Tensor^\ell(V), U \in \Tensor^m(V)$に対して$\Alt(S \otimes T \otimes U) = \Alt(\Alt(S \otimes T) \otimes U) = \Alt(S \otimes \Alt(T \otimes U))$となることを示す.$\Alt$は冪等なので,$\Alt(\Alt(S \otimes T) - S \otimes T)=0$であるから,前段の結果より$\Alt(\Alt(S \otimes T) \otimes U - S \otimes T \otimes U)=0$を得る.他の等式も同様である.
\item 命題を証明する.
\begin{align}
(\omega \wedge \eta) \wedge \theta &= \frac{(k+\ell)!}{k!\ell!}\Alt(\omega \otimes \eta) \wedge \theta \\
&= \frac{(k+\ell)!}{k!\ell!} \frac{(k+\ell+m)!}{(k+\ell)!m!} \Alt(\Alt(\omega \otimes \eta) \otimes \theta) \\
&= \frac{(k+\ell+m)!}{k!\ell!m!} \Alt(\omega \otimes \eta \otimes \theta) \\
\end{align}となる.もう一つの等式も同様にすればよい.
\end{enumerate}
\end{enumerate}
\end{proof}

外積が結合的であることを踏まえて,これ以降は3つ以上の交代テンソルの外積は括弧を省いて$\omega \wedge \eta \wedge \theta$のように書く.

\begin{dig}
交代テンソルに当てる記号として$\omega, \eta, \theta$と進めるのは Spivak にならってのことなのだが,どのような規則性に依るものなのか,何ならどれくらい一般的に使われているのか,私はよく知らない.たぶんそれっぽくてかっこいいギリシャ文字を使っているだけなのではなかろうか.
\end{dig}

\subsection{ベクトル場と微分形式}

\subsection{鎖体}

\subsection{微分形式の積分と Stokes の定理}