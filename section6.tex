\section{微分可能多様体}

微分可能多様体を定義し,前節に定義したベクトル場や微分形式,Stokes の定理を微分可能多様体にまで拡張する.微分可能多様体に対する Stokes の定理が,古典ベクトル解析における大定理たちを系として持つことを証明する.

\subsection{多様体の定義}

$k$次元多様体とは,大雑把に言ってしまえば「局所的に$\Real^k$と同一視できる空間」である.それを厳密な形で述べることからはじめる.

\begin{defi}
$M \subset \Real^n$の各点$p$が次の条件を充たすとき,$M$は$k$次元(微分可能)多様体であると言われる;$p$を含む開集合$U_p \subset \Real^n$,および開集合$V_p \subset \Real^n$ならびに微分同相写像$h_p \colon U_p \to V_p$が存在して,$h_p(U_p \cap M) = V_p \cap \left(\Real^k \times \set{0}\right)$,すなわち$h_p(U_p \cap M) = \set{y \in V_p | y^{k+1} = \dots = y^n = 0}$が成り立つ.
\end{defi}

\begin{exm}
自明な$k$次元微分可能多様体の例は$\Real^k$の開集合である.もう少し自明でない例として,$k$次元球面$\Sphere^k \coloneqq \set{p \in \Real^{k+1} | \|p\|=1}$がある.これが微分可能多様体であることを見たければ,上記の条件を直接確かめればよい.のだが,面倒なのでまだノートに書き起こしていない.
\end{exm}

ここに「面倒」と書いたが,$\Real^n$の部分集合が多様体であるかどうか,もっとかんたんに判定できる場合がある;

\begin{prop}
$A \subset \Real^n$を開集合,$g \colon A \to \Real^p$を$C^\infty$級関数とする.更に,$g(p)=0$となるような$p \in A$において,$g$の Jacobi 行列の階数が$k$であったと仮定する.このとき,$M \coloneqq g^{-1}(0)$は$n-k$次元多様体である.
\end{prop}

\begin{proof}
\cref{陰関数定理の言い換え}より,各点$p \in g(M)$に対して,$g(p)$を含む開集合$U_p$上で定義された微分同相$F_p \colon U_p \to \Real^n$であって,$g \circ {F_p}^{-1} (p) = (p^{n-k+1}, \dots, p^n)$を充たすようなものがある.$F_p(U_p) \coloneqq V_p$と置く.$F_p \colon U_p \to V_p$が,定義の条件にあるような微分同相写像であること,すなわち$F_p(U_p \cap M) = V_p \cap \left( \Real^{n-k} \times \set{0} \right)$を示せばよい.

$F_p(U_p \cap M) \subset V_p \cap \left( \Real^{n-k} \times \set{0} \right)$を示すために,$q \in U_p \cap M$を任意に取って$r \coloneqq F_p(q)$と置く.$r \in V_p$はよい.$g \circ {F_p}^{-1}(r) = g(p) = 0$なので,$r^{n-p+1} = \dots = r^n = 0$,すなわち$r \in \Real^{n-k} \times \set{0}$がわかる.

逆の包含を示すために$r \in V_p \cap \left( \Real^{n-k} \times \set{0} \right)$を取る.${F_p}^{-1}(r) \in U_p$はよい.$g \circ {F_p}^{-1}(r)=0$なので${F_p}^{-1}(r) \in M$である.
\end{proof}

この命題によって$\Sphere^n$が$n$次元多様体となることがすぐわかる.というのも,$g \colon \Real^{n+1} \to \Real$を$g(p) \coloneqq \|p\|^2 - 1$によって定めれば,$g$は$p \in \Sphere^n$において微分が消えておらず,したがってこの命題の前件を充たすとわかるからである.

さて,もう一つ定義と同値な条件を述べておこう.ベクトル場や微分形式を多様体の上で定義するには,こちらの条件のほうが使いやすい.

\begin{thm}
$M \subset \Real^n$が$k$次元多様体であることと,$M$の各点$p$が次の条件(座標条件と呼ばれる)を充たすことは同値である;$p$を含む開集合$U_p \subset \Real^n$ならびに開集合$W_p \subset \Real^k$,単射な$C^\infty$級関数$f \colon W_p \to \Real^n$が存在して,
\begin{enumerate}
\item $f(W_p) = M \cap U_p$.
\item $W_p$の各点$q$において,$f$の Jacobi 行列$J_f(q)$の階数は$k$.
\item $f^{-1} \colon f(W_p) \to W_p$は連続.
\end{enumerate}
\end{thm}

\begin{proof}
$M$が$k$次元多様体であったとする.$p \in M$を任意に取れば,定義にあるような微分同相$h_p \colon U_p \to V_p$が存在する.このとき,$W_p \coloneqq \set{(b^1, \dots, b^k) | b \in V_p \cap (\Real^k \times \set{0})}$として,$f \colon W_p \to \Real^n$を$f(a) \coloneqq {h_p}^{-1}(a,0)$で定める.$h_p$が$C^\infty$級微分同相であることから$f$の単射性と$C^\infty$級であることがわかる.$W_p \subset \Real^k$が開集合であることはよく,$(a,0) \in V_p \cap (\Real^k \times \set{0})$なので$f(a) = {h_p}^{-1}(a,0) \in U_p \cap M$である.更に$f^{-1} = h_p|_{f(W_p)}$なので(連続関数の制限は常に連続なのだから)$f^{-1}$は連続である.最後に,関数$H_p \colon \Real^n \to \Real^k$を$H_p(a) \coloneqq ({h_p}^1(a), \dots, {h_p}^k(a))$によって定めれば,$H_p$は微分可能であり,更に$H_p(f(q))=q$を充たす.したがって Jacobi 行列を考えることで$J_{H_p}(f(q)) \cdot J_f(q) = I_k$がわかるので,特に$J_f(q)$の階数は$k$である.

逆に$M$の各点が座標条件を充たすと仮定する.$p \in M$を任意に取れば,座標条件にあらわれる関数$f \colon W_p \to \Real^n$が存在する.$p = f(q)$によって$q \in \Real^k$を定める.関数$g \colon W_p \times \Real^{n-k} \to \Real^n$を,$g(a,b) \coloneqq f(a) + (0,b)$で定めると,$\det D_g(q,0) = \det D_f(q) \neq 0$であることから,逆関数定理より$(q,0)$を含む開集合$V'$ならびに$p$を含む開集合$U'$が存在して,$g$は$V'$上で$C^\infty$級の逆写像$h_p \colon U' \to V'$を持つ.また,$f$が座標条件を充たすことから,$\set{f(a) | (a,0) \in V' } = U'' \cap M$を充たす開集合$U''$が存在する\footnote{この部分は行間が大きいと思うが,本文に組み込むと議論の見通しが悪くなるので脚注で補う;座標条件の (i) より,$f(W_p) = U_1 \cap M$となるような開集合$U_1$がある.更に座標条件の (iii) より$f^{-1}$は連続であるから,$W' \coloneqq \set{ a | (a,0) \in V'}$は開集合であることとあわせれば\cref{開集合の引き戻しは開集合}より$\set{f(a) | (a,0) \in V' } = f(W') = U_2 \cap f(W_p)$となるような開集合$U_2$がある.$U'' \coloneqq U_1 \cap U_2$とすればよい.}.$U_p \coloneqq U' \cap U''$,$V_p \coloneqq g^{-1}(U_p)$とする.このとき,$U_p \cap M = \set{ f(a) | (a,0) \in V_p} = \set{ g(a,0) | (a,0) \in V_p}$なので\footnote{この部分も行間があるので補う.2つ目の等号はよいと思うので,$U_p \cap M = \set{ g(a,0) | (a,0) \in V_p}$を示す.$V_p \subset V'$なので,$ \set{ g(a,0) | (a,0) \in V_p} \subset  \set{ g(a,0) | (a,0) \in V'} = U'' \cap M$であり,かつ$\set{ g(a,0) | (a,0) \in V_p} \subset g(V') = U'$であるから,$\set{ g(a,0) | (a,0) \in V_p} \subset U' \cap U'' \cap M = U_p \cap M$である.逆の包含を示すために$p \in U_p \cap M$を取る.$p \notin \set{ g(a,0) | (a,0) \in V_p}$であったとすると,$p \in \set{ g(a,0) | (a,0) \in V'}$なので,結局$p \in \set{ g(a,0) | (a,0) \in V' \setminus V_p}$であることになるが,これは$p \in U_p = g(V_p)$に反する.},$h_p(U_p \cap M) = g^{-1}(U_p \cap M) = g^{-1}(\set{g(a,0) | (a,0) \in V_p}) = \set{(a,0) | (a,0) \in V_p} = V_p \cap (\Real^k \times \set{0})$である.
\end{proof}

ここに現れた$f \colon W_p \to \Real^n$の逆写像$f^{-1} \colon f(W_p) \to W_p$を,$p$のまわりの座標系という.今後$M$上の座標系を取る場合には,いちいち逆写像を取り直したり定義域を書き下すのが煩雑なので,「$p$のまわりの座標系$f \colon V_p \to \Real^k$」などと,逆写像を持ち出さずに記号を定義し,定義域の宣言も暗黙に済ませることにする.

\begin{exm}
上記の定理における座標条件の (iii) は本質的である.例えば単位開区間$(0,1)$を6の字につないだ図形を考えると,これは1次元多様体ではない.「T字」になっている部分において座標が取れないためである.
\end{exm}

座標条件において,座標系は単に「連続」であることしか課していない(微分可能性を課していない)ので,座標系がたかだか連続でしかない例があるのではないかと思うかもしれないが,実際にはそのようなことはない.証明を見ればわかるように座標系はある微分可能な関数の制限として得られるし,他にも次のようなことが成り立つ.

\begin{prop}
$f_1 \colon V_1 \to \Real^k$並びに$f_2 \colon V_2 \to \Real^k$を$p$のまわりの座標系とする.このとき,$f_2 \circ f_1^{-1} \colon f_1(V_1) \to \Real^k$は$C^\infty$級であり,その Jacobi 行列は正則である.
\end{prop}

\begin{proof}
$f_2$の逆写像を$f' \colon f_2(V_2) \to V_2$とする.このとき,先の定理の証明と同様にして関数$g \colon f_2(V_2) \times \Real^{n-k} \to \Real^n$ならびにその逆写像$h \colon U' \to V'$が取れる.このとき$f_2 = (h^1, \dots, h^k)$であることが以下のようにしてわかる;$a \in f_2(V_2)$を任意に取れば,$f_2 (g(a,0)) = f_2 \circ f'(a) = a$であり,かつ$h \circ g(a,0) = (a,0)$であることと逆写像の一意性から$f_2 = (h^1, \dots, h^k)$である.$f_1^{-1}$の Jacobi 行列の階数が$k$であることと,$h$が微分同相であることをあわせて結論を得る.
\end{proof}

Stokes の定理を多様体に対して拡張するにあたり,ふち付き多様体の概念を導入する必要がある.

\begin{defi}
$\Hyp^k \coloneqq \set{(p^1, \dots, p^k) \in \Real^k | p^k \geq 0}$を半空間という.$M \subset \Real^n$の各点$p$が次の条件のいずれかを充たすとき,$M$は$k$次元ふち付き多様体であると言われる.
\begin{enumerate}
\item $p$を含む開集合$U_p \subset \Real^n$,および開集合$V_p \subset \Real^n$ならびに微分同相写像$h_p \colon U_p \to V_p$が存在して,$h_p(U_p \cap M) = V_p \cap \left(\Real^k \times \set{0}\right)$,すなわち$h_p(U_p \cap M) = \set{y \in V_p | y^{k+1} = \dots = y^n = 0}$が成り立つ.
\item $p$を含む開集合$U_p \subset \Real^n$,および開集合$V_p \subset \Real^n$ならびに微分同相写像$h_p \colon U_p \to V_p$が存在して,$h_p(U_p \cap M) = V_p \cap \left(\Hyp^k \times \set{0}\right)$,すなわち$h_p(U_p \cap M) = \set{y \in V_p | y^k \geq 0, y^{k+1} = \dots = y^n = 0}$が成り立ち,更に$h_p(p)$の第$k$成分は0である.
\end{enumerate}
上の条件のうち (ii) を充たすような点全体のことを$\partial M$と書いて,$M$のふち,または境界という.定義の (ii) にある$h_p$の存在より,$\partial M$が$k-1$次元多様体であることがわかる.
\end{defi}

\begin{prop}
$M$をふち付き多様体とする.このとき,各点$p \in M$は上記の条件 (i) または (ii) のどちらかしか充たさない.言い換えると,(i) ならびに (ii) を同時に充たす$p \in M$は存在しない.
\end{prop}

\begin{proof}
背理法による.$p \in M$が (i) と (ii) を同時に充たすとして,(i) にあるような微分同相$h_1 \colon U \to V_1$ならびに (ii) にあるような微分同相$h_2 \colon U \to V_2$をとる.$g_1 \coloneqq (h_1^1,\dots, h_1^k)$,$g_2 \coloneqq (h_2^1, \dots, h_2^k)$によって$g_1 \colon U \to \Real^k$,$g_2 \colon U \to \Hyp^k$を定めると,$g_2 \circ g_1^{-1}$は$g_1(p)$を含む$\Real^k$の開集合を$\Hyp^k$の部分集合(とくに$\Real^k$の開集合ではない)にうつす.ところが$\det D_{g_2 \circ g_1^{-1}}(g_1(p)) \neq 0$なので,これは逆写像定理に反する.
\end{proof}

\subsection{多様体上のベクトル場と微分形式}

ベクトル場ならびに微分形式の定義を多様体まで拡張する.とはいえ,多様体になったことで話が変わる部分はそう大きくない.以下しばらく,$M$を$k$次元多様体とする.

\begin{defi}
$p \in M$とし,$f \colon V_p \to \Real^k$を$p$のまわりの座標系,$g \colon f(V_p) \to \Real^n$をその逆写像とする.座標系の定義より,$g$の Jacobi 行列の階数は$k$であるから,特にその微分$dg_{f(p)} \colon T_{f(p)}(f(V_p)) \to T_p \Real^n$の像は$T_p \Real^n$の$k$次元部分線型空間をなす.この像$\Image dg_{f(p)}$のことを$T_pM$と書いて,$M$の$p$における接空間という.$T_pM$の元のことを$M$の$p$における接ベクトルという.

$TM \coloneqq \bigcup_{p \in M} T_pM$を$M$の接ベクトル束という.$X \colon M \to TM$が$M$上のベクトル場であるとは,$X(p) \coloneqq X_p$が$T_pM$の元であることをいう.
\end{defi}

ベクトル場は,必ずしも$M$全域で定義されている必要はない.特に,ベクトル場として$p \in M$のまわりの座標系$y \colon V_p \to \Real^k$を考えて,$V_p$上でのみ定義されたものを考えることがある.そのようなベクトル場のことは「$p$のまわりで定義されたベクトル場」のように言うことにする.

\begin{defi}
$p \in M$とし,$y \colon V_p \to \Real^k$を$p$のまわりの座標系とする.このとき,$p$のまわりで定義されたベクトル場$\frac{\partial}{\partial y^i}$を次のように定める;$y$の逆写像を$z \colon y(V_p) \to V_p$とし,$\Real^k$に標準座標から定まるベクトル場を$\frac{\partial}{\partial x^i}$と書く.このとき,$\left( \frac{\partial}{\partial y^i} \right)_p \coloneqq dz_p \left(\frac{\partial}{\partial x^i} \right)_{y(p)}$と定める.$z$の Jacobi 行列の階数が$k$であることから,$\left\{\left(\frac{\partial}{\partial y^1}\right)_p, \dots, \left(\frac{\partial}{\partial y^k}\right)_p\right\}$は$T_pM$の基底をなす.
\end{defi}

この定義から,自動的に$\frac{\partial}{\partial y^i}$は$f \colon V_p \to \Real$に対する微分作用素となる.具体的に作用を書き下せば,
\begin{equation}
\left( \frac{\partial}{\partial y^i} \right)_p f = dz_p \left(\frac{\partial}{\partial x^i} \right)_{y(p)} f = \left(\frac{\partial}{\partial x^i} \right)_{y(p)} (f \circ z) = \partial_i  (f \circ z) (y(p)) = \partial_i  (f \circ y^{-1}) (y(p))
\end{equation}となる.この作用は,\cref{Euclid空間での座標系から定まるベクトル場}における定義と酷似していることに気づかれたい\footnote{というか,ここの定義から逆算して Euclid 空間の場合を定義した,というのが実際のところである.}.座標変換則もまた,Euclid 空間の場合と同様にして得られる.

\begin{prop}
$y=(y^1, \dots, y^n)$ならびに$z = (z^1,\dots, z^n)$を$U \subset M$上の座標系とする.このとき,
\begin{equation}
\left( \frac{\partial}{\partial z^i} \right)_p = \sum_{j} \left( \frac{\partial }{\partial z^i} y^j \right) (p) \cdot \left( \frac{\partial}{\partial y^j} \right)_p
\end{equation}が成り立つ.したがって,ベクトル場の変換則は
\begin{equation}
\frac{\partial}{\partial z^i} = \sum_{j} \frac{\partial y^j}{\partial z^i} \frac{\partial}{\partial y^j}
\end{equation}と書ける.
\end{prop}

微分形式ならびに関係する概念の定義,および基本性質たちも,Euclid 空間の場合と同様にして話が進む.あらわれる証明も Euclid 空間の場合と同様な(はずな)ので,すべて省略する.

\begin{defi}
$p \in M$に対して,$T_pM$の双対空間$(T_pM)^*$のことを$T_p^*M$と書いて,$p$における余接空間という.$T_p^*M$も線型空間の構造を持つ.
更に,$y$の定めるベクトル場は$T_pM$の基底$\left\{ \left(\frac{\partial}{\partial y^i}\right)_p,\dots, \left(\frac{\partial}{\partial y^i}\right)_p \right\}$を定めた.この基底に関する双対基底を$\left\{ (dy^1)_p,\dots, \left(dy^n\right)_p \right\}$と書く.
\end{defi}

\begin{defi}
$y \colon U \to \Real^k$を$M$上の座標系とする.自然数$k \geq 1$に対して$\omega \colon U \to \bigcup_{p \in M} \GrassAlg^k(T_p^*M)$が$U$上の微分$k$-形式,あるいは単に$k$-形式であるとは,任意の$p \in U$に対して$\omega_p \coloneqq \omega(p) \in \GrassAlg^k(T_p^*M)$となることをいう.また,(滑らかな)関数$f \colon U \to \Real$のことを微分0-形式ともいう.$U$上の微分$k$-形式全体の集合を$\DiffForm^k(U)$と書く.$\omega \in \DiffForm^k(U)$の関数倍$f\omega$のことを$f \wedge \omega$とも書く.
\end{defi}

\begin{defi}
$\omega$を$U$上の$k$-形式とする.このとき,任意の$p \in U$に対して$\left\{ \left( dy^{i_1} \right)_p \wedge \dots \wedge \left( dy^{i_k} \right)_p \ \mid \ 1 \leq i_1 < \dots < i_k \leq n \right\}$は$\bigwedge^k T^*_pM$の基底をなすので,$(i_1, \dots, i_k)$で添字付けられたある関数の族$\set{ f^y_{i_1,i_2,\dots,i_n} \colon U \to \Real | 1 \leq i_1 < \dots < i_k \leq n }$が一意的に存在して,
\begin{equation}
\omega_p = \sum_{1 \leq i_1 < i_2 < \dots i_k \leq n} {f^y_{i_1,i_2,\dots,i_n}}(p) \cdot \left( dy^{i_1} \right)_p \wedge \dots \wedge \left( dy^{i_k} \right)_p
\end{equation}が成り立つ.このような表示のことを$y$による$\omega$の(局所)座標表示と呼ぶ.$\omega$が$C^\infty$級,あるいは滑らかであるとは,すべての$f^y_{i_1,i_2,\dots,i_n}$が$C^\infty$級であることと定める.
\end{defi}

\begin{prop}
$y=(y^1, \dots, y^n)$ならびに$z = (z^1,\dots, z^n)$を$U \subset M$上の座標系とする.このとき,
\begin{equation}
\left(dz^i\right)_p = \sum_{j} \left( \frac{\partial }{\partial y^j} z^i \right) (p) \cdot \left( dy^j \right)_p
\end{equation}が成り立つ.したがって微分1-形式の変換則は
\begin{equation}
dz^i = \sum_{j} \frac{\partial z^i}{\partial y^j} dy^j
\end{equation}と書ける.
\end{prop}

\begin{prop}
$y=(y^1, \dots, y^n)$ならびに$z = (z^1,\dots, z^n)$を$U \subset M$上の座標系とする.このとき,$1 \leq i_1 < \dots < i_k \leq n$を充たすような$(i_1, \dots, i_k)$に対して
\begin{equation}
\left(dy^{i_1} \right)_p \wedge \dots \wedge \left(dy^{i_k} \right)_p = \sum_{1 \leq j_1 < \dots < j_k \leq n} \det \left( \frac{\partial y^{i_a}}{\partial z^{j_b}} \right) (p) \cdot \left( dz^{j_1} \right)_p \wedge \dots \wedge \left( dz^{j_k} \right)_p
\end{equation}が成り立つ.但し,$\left( \frac{\partial y^{i_a}}{\partial z^{j_b}} \right)$は$(a,b)$成分が$\frac{\partial y^{i_a}}{\partial z^{j_b}}$であるような$k \times k$行列をあらわす.特に$n$-形式の場合は記述が少し簡単になり,
\begin{equation}
\left(dy^1 \right)_p \wedge \dots \wedge \left(dy^n \right)_p = \det \left( \frac{\partial y^i}{\partial z^j} \right) (p) \cdot \left( dz^1 \right)_p \wedge \dots \wedge \left( dz^n \right)_p
\end{equation}が成り立つ.
\end{prop}

\begin{defi}$\omega \in \DiffForm^n(U)$とし,その座標表示
\begin{equation}
\omega_p = \sum_{1 \leq i_1 < i_2 < \dots < i_k \leq n} {f^y_{i_1,i_2,\dots,i_k}}(p) \cdot \left( dy^{i_1} \right)_p \wedge \dots \wedge \left( dy^{i_k} \right)_p
\end{equation}を考える.このもとで,$\omega$の外微分$d \omega \in \DiffForm^{n+1}(U)$を,
\begin{equation}
d\omega_p \coloneqq \sum_{1 \leq i_1 < i_2 < \dots < i_k \leq n} \left(df^y_{i_1,i_2,\dots,i_k}\right)_p \wedge \left( dy^{i_1} \right)_p \wedge \dots \wedge \left( dy^{i_k} \right)_p
\end{equation}
によって定める.
\end{defi}

\begin{prop}[外微分の基本性質] $\omega \in \DiffForm^k(U)$,$\eta \in \DiffForm^\ell(U)$とする.
\begin{enumerate}
\item $k=\ell$のとき,$\dif (\omega + \eta) = \dif \omega + \dif \eta$.
\item $\dif (\omega \wedge \eta) = (\dif \omega) \wedge \eta + (-1)^k \omega \wedge \dif \eta$.
\item $\dif (\dif \omega) = 0$.$\dif^2 = 0$と書く場合も多い.
\end{enumerate}
\end{prop}

\begin{defi}$M$,$N$を多様体とし,$y \colon U \to \Real^k$を$N$の座標系とする.このとき,$\omega \in \DiffForm^k(U)$の$f \colon M \to N$による引き戻し$f^* \omega \in \DiffForm^k(f^{-1}(U))$を,$v_1, \dots, v_k \in T_pM$に対して
\begin{equation}
(f^* \omega)_p (v_1, \dots, v_k) \coloneqq \omega_{f(p)} (\dif f_p v_1, \dots, \dif f_p v_k)
\end{equation}とすることで定める.
\end{defi}

\begin{prop}[引き戻しの基本性質] $\omega \in \DiffForm^k(U)$,$\eta \in \DiffForm^\ell(U)$,$U \subset N$を開集合とし,$f \colon M \to N$とする.
\begin{enumerate}
\item $k = \ell$のとき,$f^*(\omega + \eta) = f^* \omega + f^* \eta$.
\item $f^*(\omega \wedge \eta) = f^* \omega \wedge f^* \eta$.
\item $f^*(\dif \omega) = \dif (f^*(\omega))$.
\end{enumerate}
\end{prop}

\subsection{多様体の向き}

微分形式の座標変換則には Jacobi 行列の行列式が現れる.それが原因で,座標の取り替え方によっては符号が変わる.なので,Euclid 空間において微分形式の積分を well-defined に定義するには,座標変換の Jacobi 行列の行列式が正であることを要求しなければいけなかった.それと同様の要求が微分可能多様体に対しても必要になる.それが「向きを選ぶ」という操作である.

\begin{defi}
線型空間$V$の基底上の二項関係$\sim$を次のように定める;「$\mathcal{V} = \set{v_1, \dots, v_n}$ならびに$\mathcal{W} = \set{w_1, \dots, w_n}$を$V$の基底とし,$w_i \eqqcolon \sum a_i^j v_j$と置く.$\mathcal{V} \sim \mathcal{W}$となるのは,$\det \left(a_i^j\right) > 0$がなりたつとき」.$\sim$は$V$の基底に対する同値関係を定める.この同値関係に関する同値類のことを$V$の向きという.$\set{v_1, \dots, v_n}$が属する向きのことを$[ v_1, \dots, v_n ]$と書き,属さない向きのことを$- [ v_1, \dots, v_n ]$と書く.特に$V=\Real^n$の場合,標準基底$\set{e_1, \dots, e_n}$が属する向き$[e_1, \dots, e_n]$を自然な向き,または右手系といい,右手系でない向きを左手系という.
\end{defi}

\begin{que}
$V$の向きは常に2つ存在する.
\end{que}

\begin{dig}
このノートではここで「右手系」や「左手系」を定義した.ここより前でこれらの言葉を証明に組み込むことはしていない.が,それはそれとして,$\Real^2$や$\Real^3$の場合に「右手系」や「左手系」という言葉を多少なりとも見た人は多いと思う.
\end{dig}

\begin{defi}
$M$を$k$次元多様体とする.各$p \in M$ごとに,$T_pM$の向きのひとつ$\mu_p$を選んで集めた集合$\set{\mu_p | p \in M}$のことを,$M$の接空間の向きの系という.

$\set{\mu_p | p \in M}$を$M$の接空間の向きの系とする.向きの系$\set{\mu_p | p \in M}$は次の条件を充たすときに一貫しているといわれ,充たさないときに一貫してないといわれる;任意の座標系$f \colon V_p \to \Real^k$をとり,その逆写像を$g \colon f(V_p) \to V_p$とするとき,任意の$a,b \in f(V_p)$に対して$[dg_a(\Tan_a(e_1)), dg_a(\Tan_a(e_2)), \dots, dg_a(\Tan_a(e_k))] = \mu_{g(a)}$と$[dg_b(\Tan_b(e_1)), dg_b(\Tan_b(e_2)), \dots, dg_b(\Tan_b(e_k))] = \mu_{g(b)}$が同値である.

一貫した$M$の接空間の向きの系が存在するとき,$M$は向き付け可能であるという.このとき,ふたつ存在する一貫した$M$の接空間の向きの系のおのおのを$M$の向きという.向きの定まった多様体のことを向きづけられた多様体という.
\end{defi}

\begin{defi}
$M$を向き付けられた多様体とし,その向きを$\mu = \set{\mu_p | p \in M}$とする.座標系$f \colon V_p \to \Real^k$をとり,その逆写像を$g \colon f(V_p) \to V_p$とする.$f$が向きを保つとは,ある$a \in f(V_p)$に対して,すなわち任意の$a \in f(V_p)$に対して$[dg_a(\Tan_a(e_1)), \dots, dg_a(\Tan_a(e_k))] = \mu_{g(a)}$が成り立つことをいう.
\end{defi}

$f$が向きを保たないならば,線型変換$T \colon \Real^k \to \Real^k$であって$\det T = -1$であるようなものを取れば$T \circ f$は向きを保つ.したがって,各点において向きを保つ座標系が存在する.

\begin{lem}
$f_1, f_2 \colon V_p \to \Real^k$を$p$のまわりの向きを保つ座標系とする.このとき,$\det D_{f_2 \circ f_1^{-1}} > 0$である.
\end{lem}

\begin{proof}$f_1$,$f_2$の逆写像をそれぞれ$g_1$,$g_2$とする.$f_1$と$f_2$がいずれも向きを保つことから,
\begin{equation}
[d{g_1}_p(\Tan_p(e_1)), d{g_1}_p(\Tan_p(e_2)), \dots, d{g_1}_p(\Tan_p(e_k))] = [d{g_2}_p(\Tan_p(e_1)), d{g_2}_p(\Tan_p(e_2)), \dots, d{g_2}_p(\Tan_p(e_k))]
\end{equation}
である.更に$g_2 \circ f_2 \circ g_1 = g_1$なので,特に$dg_2 \circ d(f_2 \circ g_1) = dg_1$を得る.したがって
\begin{align}
& [dg_2 \circ d(f_2 \circ g_1)_p(\Tan_p(e_1)), dg_2 \circ d(f_2 \circ g_1)_p(\Tan_p(e_2)), \dots, dg_2 \circ d(f_2 \circ g_1)_p(\Tan_p(e_k))] \\
= &[d{g_2}_p(\Tan_p(e_1)), d{g_2}_p(\Tan_p(e_2)), \dots, d{g_2}_p(\Tan_p(e_k))]
\end{align}を得るから,
\begin{align}
& [d(f_2 \circ g_1)_p(\Tan_p(e_1)), d(f_2 \circ g_1)_p(\Tan_p(e_2)), \dots, d(f_2 \circ g_1)_p(\Tan_p(e_k))] \\
= &[\Tan_p(e_1), \Tan_p(e_2), \Tan_p(e_k)]
\end{align}となる.したがって$\det D_{f_2 \circ f_1^{-1}} > 0$を得る.
\end{proof}

\begin{defi}
$M$を$k$次元ふち付き多様体とし,$p \in \partial M$ならびに$p$のまわりの座標系$f \colon V_p \to \Hyp^k$を任意に取る.このとき$T_p \partial M$は$T_pM$の$k-1$次元部分線型空間である.したがって,$T_pM$における$T_p \partial M$の直交補空間は1次元線型空間をなすから,特にこの直交補空間に属する長さ1のベクトル$v$が2本取れる.その中で,$df_p(v)$の$\frac{\partial}{\partial x^k}$成分が負のものを,$M$の$p$における単位外法線といい$n(p)$であらわす.
\end{defi}

\begin{defi}
$M$を$k$次元ふち付き多様体とし,$\mu$を$M$の向きとする.$p \in \partial M$を任意に取る.このとき,$v_1, \dots, v_{k-1} \in T_p \partial M$を,$[n(p), v_1, \dots, v_{k-1}] = \mu_p$が成り立つように取る.ここで,$v_1, \dots, v_{k-1}$が定める$T_p \partial M$の向き$[v_1, \dots, v_{k-1}]$を$(\partial M)_p$と書く.$\set{(\partial M)_p | p \in M}$は$\partial M$の向きを定める.この向きを$M$から定まる$\partial M$の向きという.
\end{defi}

\subsection{Stokes の定理}

\subsection{Riemann 計量と体積形式}

\cref{行列式の特徴づけ}を踏まえれば,行列式$\det$の定義として,$\omega \in \GrassAlg^n(\Real^n)$の元であって$\omega(e_1, \dots, e_n) = 1$を充たすもの,という定義を採っても構わない\footnote{学部の線型代数の講義でいきなり外積代数の一般論を広げてこの定義をするのはとっつきづらくて敬遠される気もする.のだが,同じくらいとっつきにくい(と私は感じる)Leibniz の明示公式は学部1年で教わるのだし,やり方を工夫して外積代数の一般論を学部1年生に仕込めたりしないのだろうか?}.この定義は一般の線型空間$V$に対してそのままは通らないが,内積が与えられているもとで一般化した概念を定義することはできる;

\begin{thm}
$T$を$V$の内積とし,$\set{v_1, \dots, v_n}$を$T$に関する正規直交基底とする.このとき,ある$\omega \in \GrassAlg^n(V^*)$が存在して,$[v_1, \dots, v_n] = [w_1, \dots, w_n]$を充たすような任意の正規直交基底$\set{w_1, \dots, w_n}$に対し$\omega (v_1, \dots, v_n) = 1$を充たす.この$\omega$を,内積$T$および向き$[v_1, \dots, v_n]$の定める$V$の体積要素という.特に,$\Real^n$の標準内積および自然な向きの定める体積要素は行列式$\det$である.
\end{thm}

\begin{proof}
$\set{v_1^*, \dots, v_n^*}$を$\set{v_1, \dots ,v_n}$の双対基底とすれば,$\omega \coloneqq v_1^* \wedge \dots \wedge v_n^*$は$\omega(v_1, \dots ,v_n) = 1$を充たす.$w_i \eqqcolon \sum a_i^j v_j$によって行列$A \eqqcolon \left(a_i^j\right)$を定めれば,$\Real^n$の場合と同様の議論で$A$は直交行列になることがわかるので,$\det A = \pm 1$となる.したがって\cref{最高次交代テンソルの変換則}により,$[v_1, \dots, v_n] = [w_1, \dots, w_n]$であるならば$\omega(w_1, \dots, w_n) = 1$である.
\end{proof}

\begin{prop}
$V = \Real^n$とする.$v_1, \dots, v_{n-1} \in V$を任意にとって固定する.このとき,任意の$w \in V$に対して,次の式を充たすような$z \in V$が一意的に存在する;
\begin{equation}
\langle z,w \rangle = \det\, (v_1, \dots, v_{n-1}, w).
\end{equation}この$z$を$v_1 \times \dots \times v_{n-1}$と書いて,$v_1, \dots, v_{n-1}$のクロス積またはベクトル積という.
\end{prop}

\begin{proof}
$\varphi \colon V \to \Real$を$\varphi(w) \coloneqq \det\, (v_1, \dots, v_{n-1}, w)$で定めれば,$\varphi$は線型写像なので,$\varphi \in V^*$である.したがって\cref{双対空間は縦ベクトル}の結果より任意の$w \in V$に対して$\varphi(w) = \langle z,w \rangle$を充たす$z$が一意的に存在する.
\end{proof}

このクロス積は,背伸びした高校生が学んだり,学部のベクトル解析で見かけるであろうあのクロス積と実際には同じものである.のだが,定義だけをみてもそれがわかる気がしないので,クロス積の諸性質(こちらのほうがまだ見慣れているだろう)を証明しておくことにする;

\begin{prop}[クロス積の基本性質]$V = \Real^n$とする.
\begin{enumerate}
\item クロス積をとる写像$V^{n-1} \ni (v_1, \dots, v_{n-1}) \mapsto v_1 \times \dots \times v_{n-1} \in V$は多重線型である.
\item 任意の$\sigma \in \Sym_{n-1}$に対して,$v_{\sigma(1)} \times \dots \times v_{\sigma(n-1)} = \sgn \sigma (v_1 \times \dots \times v_{n-1})$.
\item 以下$V=\Real^3$とする.$x = \sum_i x^i e_i$ならびに$y = \sum_i y^i e_i$に対して,$x \times y = (x^2y^3 - x^3y^2)e_1 + (x^3y^1 - x^1y^3)e_2 + (x^1y^2 - x^2y^1)e_3$が成り立つ.
\item $\langle x, x \times y \rangle = 0$ならびに$\langle y, x \times y \rangle = 0$である.
\item $\theta$を$x$と$y$のなす角とするとき,$\|x \times y\| = \|x\| \cdot \|y\| \cdot \lvert \sin \theta \rvert$.
\end{enumerate}
\end{prop}

\begin{proof}
\leavevmode
\begin{enumerate}
\item 行列式の多重線型性より従う.
\item 行列式の交代性より従う.
\item Sarrus の公式より$\det \, (x,y,z) = (x^2y^3 - x^3y^2)z^1 + (x^3y^1 - x^1y^3)z^2 + (x^1y^2 - x^2y^1)z^3$であるからよい.
\item 前段の結果と合わせて内積を直接計算せよ.
\item $\det \, (x,y,z)$は$x,y,z \in \Real^3$の張る平行六面体の体積であった.ここで,$z = x \times y / \| x \times y \|$の場合を考えると,$\lvert \det \, (x,y,z) \rvert = \|x \times y\|$である.$\langle x, x \times y \rangle = \langle y, x \times y \rangle = 0$をすでに見たので,$\|x \times y\|$は$x$と$y$が張る平行四辺形の面積である.
\end{enumerate}
\end{proof}

後々でやるかもしれないことを見越して,クロス積に関係するようなしないようなコメントをいくつか付け加えておくことにする\footnote{実のところ,聞きかじりに基づいて「ここでこのコメントを入れておくと後で話のネタが増えるだろう」とヤマを張っているだけである.立てたフラグを回収する保証はない.};

\begin{exm}
$V = \Real^3$とする.$z \in \Real^3$に対して,$\varphi_z(x,y) \coloneqq \det\, (x,y,z)$によって関数$\varphi_z$を定めると,$\varphi_z \in \GrassAlg^2(\Real^3)$である.直接計算によって,$\varphi_{e_1} = e_2^* \wedge e_3^*$,$\varphi_{e_2} = e_3^* \wedge e_1^*$,$\varphi_{e_3} = e_1^* \wedge e_2^*$がわかるので,$\varphi \colon \Real^3 \in z \mapsto \varphi_z \in \GrassAlg^2(\Real^3)$は同型写像である.この同型は,微分2-形式をベクトル場と同一視する際に用いられる.この仕方の同一視に基づいて得られるベクトル場は物理学において軸性ベクトル場と呼ばれるものであり,通常のベクトル場(物理学では極性ベクトル場という)とは座標変換の際に受ける変換が異なる,ということを気が向いたら見る.古典電磁気学においては,電場は極性ベクトル場であり,磁束密度は軸性ベクトル場である,らしい.
\end{exm}

\begin{exm}
更に,$z = z^1 e_1 + z^2 e_2 + z^3 e_3$に対して$z^* \in \GrassAlg^1(\Real^3)$を,$z^* \coloneqq z^1 e_1^* + z^2 e_2^* + z^3 e_3^*$によって定める.この写像も同型写像であるから,この2つの同型写像を合成して得られる$\GrassAlg^1(\Real^3) \ni z^* \mapsto \varphi_z \in \GrassAlg^2(\Real^3)$も同型写像である.かくして得られた同型写像を$* \colon \GrassAlg^1(\Real^3) \to \GrassAlg^2(\Real^3)$と書いて,Hodge スター作用素という.Hodge スター作用素は微分可能多様体上の Laplacian の定義に直接的に現れるのだが,どうして出てくるのか私はよく納得していない.なのでここで寄り道して定義を出した(が,Laplacian について書くかどうかは未定である).
\end{exm}


\subsection{古典ベクトル解析の諸定理}
